% --------------------------------
% TEACHING
% --------------------------------


\vspace{0.2cm}

\mediumtitle{Teaching activity}


\smalltitle{In English}

\vspace{0.2cm}

\noindent \textbf{To PhD students, postdoctoral researchers, and PIs}

\begin{singletablelist}
	
	\stlist{2025 -- present}{\textbf{Lecturer} for the \emph{Collaborative tools for reproducible research} workshop for the PhD program in Complex Systems for Quantitative Biomedicine, at the University of Turin. The workshop (8h) is centred around the idea of ``live coding'', where attendees code along with the instructors thus getting useful hands-on experience and improving their ability to explore the topics on their own. \\
		& \textbf{Lecturer} for the \emph{Machine Learning in clinical epidemiology and pre-clinical research} workshop for the PhD program in Complex Systems for Quantitative Biomedicine, at the University of Turin. The workshop includes frontal lessons (12h) and a hands-on session (8h, lead by Alessia Visconti).\\
		&
	}
	
	\stlist{2019 -- 2023}{\textbf{Instructor} for several programming and data analysis workshops:\\
		& \hskip1cm - The Unix Shell\\
		& \hskip1cm - Version Control and collaboration with Git/GitHub\\
		& \hskip1cm - Python Programming\\
		& \hskip1cm - R Programming\\
		& \hskip1cm - Introduction to Working with Data\\
		& \hskip1cm - OpenRefine\\
		& These workshops are designed for PhD students and early career researchers but are open to researchers at every level, including PIs, are offered regularly (roughly twice a year), and have always been evaluated as ``excellent'' or ``exceptional'' by the attendees (details on some of the workshops can be found at \url{https://kcl-carpentries.github.io/}). \\
		& The workshops are centred around the idea of ``live coding'', where attendees code along with the instructors thus getting useful hands-on experience and improving their ability to explore the topics on their own.\\
		& 
	}				
	
	\stlist{2016 - 2023}{\textbf{Co-organiser} of the Regulatory Genomics journal club at the Department of Twins Research \& Genetic Epidemiology, King's College London. The journal club discusses papers on the latest achievements and methods in the fields of regulation of gene expression, epigenetics, splicing, evolution, and related topics. \\
		&
	}
		
	\stlist{2021 - 2022}{\textbf{Instructor} for \emph{The Unix Shell} and the \emph{Version Control and collaboration with Git/GitHub} workshops for students and researchers of the MRC Gambia and the AKU Nairobi Research Centres. Both workshops were rated as ``exceptional'' by the attendees.  \\
	 	&\textbf{Instructor} for the \emph{Metagenomics Data Analysis: Investigating the invisible world of microbes} (details at \url{https://alesssia.github.io/metagenomic_workshop/}). The workshop, which included frontal lessons and hands-on sessions, was designed for researchers at every level (including but not limited to PhD students, post-doctoral researchers and PIs) without any previous knowledge of the topics and tools presented. \\
		&
	}


\end{singletablelist}

% \vspace{0.2cm}

\noindent \textbf{To graduate students}

\begin{singletablelist}

	\stlist{2025 -- present}{ \textbf{Co-lecturer} for the \emph{Leveraging Large Language Models for hands-on learning in medical education and research} elective teaching activities for the \emph{Faculty of Medicine and Surgery - MedInTO}, Department of Clinical and Biological Science,  University of Turin. The workshop (10h) is organised as a set of lectures, hands-on activity, and discussions and explains what Large Language Models are, how they work, and how to use them in a critical way.\\
			&
		}

\end{singletablelist}

% \vspace{0.2cm}

\noindent \textbf{To postgraduate students}

\begin{singletablelist}

	\stlist{2013 - 2014}{\textbf{Teaching assistant} for the \emph{``Human Molecular Genetics''} MSc Department of Genomics of Common Diseases, Imperial College London. Alessia Visconti was offering support during the practical sessions on R programming as well as one-to-one meetings with students attending the following courses:\\
		%& \hskip0.5cm Workshops on: \\
		& \hskip1cm - The Unix Shell\\
		& \hskip1cm - R Programming\\
		& \hskip1cm - Exploratory Data Analysis and Probability\\
		& \hskip1cm - Quantitative genetics\\
		& \hskip1cm - Next Generation Sequencing Data Analysis.\\
		&
	}				

\end{singletablelist}

% \vspace{0.2cm}

\noindent \textbf{To medical residents}

\begin{singletablelist}

	\stlist{2025 -- present}{ \textbf{Co-lecturer} for the \emph{Integrating Large Language Models into clinical work and research: a critical approach} workshop for medical residents at University of Turin,  University of Turin. The workshop (14h) is organised as a set of lectures, hands-on activity, and discussions and explains what Large Language Models are, how they work, and how to use them in a critical way in the clinic and in clinical reasearch.\\
			&
		}

\end{singletablelist}

% \vspace{0.2cm}

\newpage

\noindent \textbf{Outside academia}


\begin{singletablelist}


\stlist{2021 - 2022}{ \textbf{Instructor} for a \emph{Version Control and collaboration with Git/GitHub} workshop at the UK Health Security Agency (UKHSA)}

\end{singletablelist}


\smalltitle{In Italian}

\vspace{0.2cm}



\noindent \textbf{To undergraduate students}


\begin{singletablelist}
	
	\stlist{2024 - present}{\textbf{Lecturer} for the \emph{``Medical statistics''} module within the \emph{``Evidence-based nursing''} course, Nursing School, Department of Clinical and Biological Science, University of Turin. Alessia Visconti was the sole responsible for the module, which covers the basics of biostatistics. The module is run in 2 parallel sessions of 15h each.\\
		&\textbf{Lecturer} for the \emph{``Medical statistics''} module within the \emph{``Preparatory to research''} course, 
Degree in Psychiatric Rehabilitation Techniques, Department of Clinical and Biological Science, University of Turin. Alessia Visconti was the sole responsible for the module, which covers the basics biostatistics and the basics of R programming and data analysis (24h).\\
		&\textbf{Co-lecturer} for the \emph{``Methodology of social-health research''} course, Degree in Professional Education, University of Turin. Alessia Visconti was the responsible for the part of the course covering the basics biostatistics (20h).\\
		&
	}
			
	\stlist{2013 - 2014}{\textbf{Teaching assistant} for the \emph{``Data analysis''} course, Department of Biological Science, University of Turin. Alessia Visconti was offering support during the practical sessions on R programming as well as one-to-one meetings with students. \\
		&\textbf{Teaching assistant} for the \emph{``Operating System''} course, Department of Computer Science, University of Turin. Alessia Visconti was offering support during the practical sessions on R programming as well as one-to-one meetings with students.\\
		&
	}
	
	\stlist{2012 - 2013}{\textbf{Lecturer} for the \emph{``Operating System and Networking''} course, Interfaculty School of Strategic Studies, University of Turin. Alessia Visconti was the sole responsible for the practical sessions covering the basis of GNU/Linux, the Unix shell, and process management. She designed the final project, \emph{i.e.}, the development of a basic client/server application in C (details at \url{https://alesssia.github.io/sistemi_elab_info_I_2012_13})\\
		&\textbf{Teaching assistant} for the \emph{``Operating System''} course,  Department of Computer Science, University of Turin. Alessia Visconti was offering support during the practical sessions as well as one-to-one meetings with students. \\
		&
	}
	
	\stlist{2011 - 2012}{\textbf{Teaching assistant} for the \emph{``Database''} course, Department of Computer Science, University of Turin. Alessia Visconti was offering support during the practical sessions as well as one-to-one meetings with students.\\
		&
	}

% \end{singletablelist}
%
% \newpage
%
% \begin{singletablelist}
			
	\stlist{2010 - 2011}{\textbf{Teaching assistant} for the \emph{``Database''} course, Department of Computer Science, University of Turin. Alessia Visconti was offering support during the practical sessions as well as one-to-one meetings with students. \\
		&\textbf{Teaching assistant} for the \emph{``Formal Language''} course, Department of Computer Science, University of Turin. Alessia Visconti was offering support during the practical sessions as well as one-to-one meetings with students. \\
		&\textbf{Teaching assistant} for the \emph{``Statistics and data mining with SAS''} course, Department of Mathematics, University of Turin. Alessia Visconti prepared slides and recorded 3 hours of video lessons on SAS Enterprise Miner. She also prepared a self-evaluation questionnaire for the students.\\
		&
	}
		
	\stlist{2009 - 2010}{\textbf{Lecturer} for the \emph{``Computer Science''} course, Department of Letters and Philosophy, University of Turin. Alessia Visconti was the sole responsible for the practical sessions covering the MS Office suite and the students' evaluation (details at \url{https://alesssia.github.io/lab_lettere_2009_10/}).\\
		&
	}
	
	\stlist{2006 - 2007}{\textbf{Teaching assistant} for the \emph{``Program Languages - JAVA''} course, Department of Computer Science, University of Turin. Alessia Visconti was offering support during the practical sessions. \\
		&
	}
	
	\stlist{2005 - 2006}{\textbf{Teaching assistant} for the \emph{``Program Languages - JAVA''} course, Department of Computer Science, University of Turin. Alessia Visconti was offering support during the practical sessions.\\
		&
	}
	
	\stlist{2004 - 2005}{\textbf{Teaching assistant}  for the \emph{``Program Languages - C''} course, Department of Computer Science, University of Turin. Alessia Visconti was offering support during the practical sessions.\\
		&
	}

\end{singletablelist}

\noindent \textbf{To medical residents}

\begin{singletablelist}
	
	\stlist{2023 - present}{\textbf{Lecturer} for \emph{Introduction to Statistics} short course for medical residents at University of Turin (9h). Alessia Visconti was the sole responsible for the module, which covers the basics of biostatistics.}

\end{singletablelist}