\mediumtitle{Awards}

\begin{doubletablelist}

	\dtlist{Summer 2019}{\textbf{Awarded a mini-grant} (£1000) to hire an undergraduate student through the ``King's Undergraduate Research Fellowship (KURF)''.\\
	& }

	\dtlist{Summer 2018}{\textbf{Awarded a mini-grant} (£1000) to hire an undergraduate student through the ``King's Undergraduate Research Fellowship (KURF)''. \\
	& }
	
	\dtlist{July 2018}{\textbf{Winner of one of the challenges} of the \emph{``BioDataHack 2018 -- Genomic, Biodata and Improving Health Outcomes''}. The project presented by Alessia Visconti and the other members of her team (Jun Aruga, Oliver Giles, Ioannis Valasakis e Chen Zhang) advanced the vision of a device that will allow the constant monitoring of IBD by patients from the comfort of their own homes, ranked first on the ARM, Cavium, and Atos Challenge: \emph{How can we use mobile technology to transform biological data processing?}. During the two-day BioData Hackathon, the team successfully ported the metagenomics pipeline developed by Alessia Visconti~\cite{Vis18b} onto Arm’s 64-bit architecture, where the data could be processed in a few hours, showing that the analysis of microbial data can be successfully taken out of centralised data centres. \\
	& The solution also implemented a neural network that, receiving as input the microbial profile produced by the analysis pipeline, could predict the disease status. Even with a very small set of training data, the proposed approach was able to separate patients with Crohn's disease and ulcerative colitis (the two main forms of IBD) from healthy controls with over 90\% accuracy. \\
	& The team also proposed a derivative score from the neural network’s output. Such a score, which could be tracked over time, would allow individuals to measure the effects of lifestyle/medication changes on their disease’s progression to assess their efficacy in almost real-time. The next step would be to use the data generated in this monitoring process to craft suggestive models, able to offer individuals advice and treatments based on what has been effective in patients with similar microbiome profiles.\\
	& }

	\dtlist{Summer 2017}{\textbf{Awarded a mini-grant} (£1000) to hire an undergraduate student through the ``King's Undergraduate Research Fellowship (KURF)''. \\
	& }

    \dtlist{Jan 2012 - Dec 2013}{\textbf{Awarded two years postdoctoral fellowship} awarded by the Italian Minister of Education, University and Research. \\
	& }

	\dtlist{Jan 2012 - Dec 2012}{\textbf{Awarded a postdoctoral training grant} (\euro{3000}) awarded by the Regione Piemonte. \\
	& }

	\dtlist{Summer 2011}{\textbf{Participation} to the \emph{``DREAM6 -- Promoter Activity Prediction Challenge''}. The proposed approach (developed with Ali Altıntas and Chris Workman) which combined results from two well-known machine learning approaches (regression trees and support vector machines for regression), ranked 8th out of 21 participants. \\
	& }

	\dtlist{Summer 2010}{ \textbf{Winner of one of the challenges} of the \emph{``DREAM5 -- Network Inference Challenge''}. The ensemble approach~\cite{Vis11b,Mar12} developed with the other team members (Roberto Esposito and Francesca Cordero), which was compared with other 35 approaches and which combines multiples approaches within a Naive Bayes classifier, despite its simplicity, ranked third for the reverse engineering of real organisms' gene regulatory networks and first for the reconstruction of the \emph{Saccharomyces Cerevisiae}'s network. \\
	& }

	\dtlist{Jan 2009 - Dec 2011}{\textbf{Awarded three years PhD fellowship} awarded by the Italian Minister of Education, University and Research (best PhD project).\\
	& }
\end{doubletablelist}



