% --------------------------------
% RESEARCH
% --------------------------------


\mediumtitle{Research posts}

\begin{doubletablelist}
	\dtlist{Dec 2023 - Present}{\textbf{Assistant professor (tenure track)} at the Center for Biostatistics, Epidemiology and Public Health, Department of Clinical and Biological Sciences, University of Turin, Italy \\	
	 &\textsc{Research activity}: Alessia Visconti is responsible for \emph{(a)}~the development and application of statistical and computational multi\emph{-omics} approaches to understand the mechanisms and to improve the diagnosis and treatment of human diseases, \emph{(b)}~the design and deployment of Deep Learning approaches for monitoring and improving public health responses~\cite{Con24}, \emph{(c)}~the teaching of under- and post-graduates courses, and \emph{(d)} the preparation of research grants, of which she has also the scientific responsibility. She has also an interest in the integration of patients' preferences in medical care and drug development~\cite{DiB24}.\\
	 & Alessia Visconti remains lead the microbiome data analysis at the Department of Twin Research \& Genetic Epidemiology, King's College London, UK, where she explores the effect of the gut microbiome on human health~\cite{Dan24,Att24,Val24,Sar25}, with a particular interest on its phage component~\cite{Kir24,Zol24}, and continues her research on IgA glycosylation~\cite{Vis24} and in the field of genetic epidemiology~\cite{Ros24,Sta25}. \\
	& 
	}
	
	\dtlist{Oct 2023 - Nov 2023}{\textbf{Senior bioinformatician} at the Genomics Research Centre, Human Technopole, Italy \\	
	 &\textsc{Research activity}: Alessia Visconti was involved in a project performing genome-wide association study (GWAS) on phenotypes acquired from cardiac MRIs \emph{via} an unsupervised deep-learning approach~\cite{Ome24}. \\
	 & 
	}
    
	\dtlist{Aug 2022 - Apr 2023}{\textbf{Career break}: Maternity leave \\
	 & 
	}
	
	\dtlist{Aug 2017 - Sep 2023}{\textbf{Research fellow} at the Department of Twin Research \& Genetic Epidemiology, King's College London, UK \\	
	 &\textsc{Research activity}: Alessia Visconti was responsible for \emph{(a)}~the development and application of statistical and computational multi\emph{-omics} approaches to understand the mechanisms and to improve the diagnosis and treatment of human diseases, \emph{(b)}~the supervision of master and PhD students as well as postdocs, \emph{(c)} the preparation of research grants, of which she has also the scientific responsibility\\
	 & From her previous role, Alessia Visconti continued to lead the bioinformatics analyses for a set of studies aiming at dissecting the aetiology of melanoma and its risk phenotypes~\cite{Vis19a,Vis20,Lan20,San20} and at studying IgA Nephropathy~\cite{Dot21}. She was also responsible for several new projects aiming at investigating, among the others, the influence of the gut microbiome on human health~\cite{Vis19,Bar20,LeR22,Zha22,Val23,Lou23,Nog23a,Nog23b}, and predicting melanoma response to immunotherapy~\cite{Ros22,Vis23}, and at studying thyroid diseases~\cite{Mar20}, atopic dermatitis~\cite{Gro21,Bud22}, cardiovascular diseases and their risk factors~\cite{Ros21}, immune system modifications~\cite{Pia21}, and X-inactivation~\cite{Zit23}. During the SARS-CoV-2 pandemic, she \emph{(a)}~developed the analysis pipeline for the daily data provided by more than four million users~\cite{Mur21}, \emph{(b)}~led two studies investigating skin manifestations of SARS-CoV-2~\cite{Vis21, Vis22}, and \emph{(c)}~performed the bioinformatics analysis for several other studies~\cite{Men20,Lee20,Zaz20,Hop21,Wil21,Sud21}. \\
	 & 
	% &\textsc{Supervisor}: Dr. Mario Falchi
	}
	
	
    \dtlist{Jun 2016 - Jun 2019}{\textbf{Honorary research associate} at CERN OpenLab, Switzerland \\
	 &\textsc{Research activity}: Alessia Visconti extended the ROOT library (\url{https://root.cern/}), developed by the CERN OpenLab for enabling the storage and scientific analyses and visualisation of large amounts of data from particle physics experiments, to allow the efficient storage of genomic data. \\
	 & 
	% &\textsc{Supervisor}: Dr. Marco Manca
	}
    
	
	\dtlist{Apr 2015 - Jul 2017}{\textbf{Research associate} at the Department of Twin Research \& Genetic Epidemiology, King's College London, UK \\ 
	 &\textsc{Research activity}: Alessia Visconti was mostly responsible for the development and application of statistical and computational multi\emph{-omics} approaches to understand the mechanisms and to improve the diagnosis and treatment of human diseases. She also supervised master and PhD students and contributed to the writing of research grants. In this role, Alessia Visconti led the bioinformatics analyses for a set of projects aiming at dissecting the aetiology of melanoma, melanoma risk phenotypes, and their connection with ageing~\cite{Rib16,Hys18,Vis18a,Duf17}, and at studying IgA Nephropathy~\cite{Lom16}, cognition and neurodevelopmental disease~\cite{Cul18}, and epigenetic modification~\cite{Zag18}. \\
	 &  She also reported on how to conduct metagenomic studies in microbiology and clinical research~\cite{Vis18c} and developed a novel pipeline which ensures reproducibility in metagenomics research~\cite{Vis18b}.\\
	 & 
	% &\textsc{Supervisor}: Dr. Mario Falchi
	}
	
    \dtlist{Jan 2014 - Mar 2015}{\textbf{Research associate} at the Department of Genomics of Common Disease, School of Public Health, Imperial College London, UK  \\
	 &\textsc{Research activity}: Alessia Visconti \emph{(a)}~developed and implemented a novel approach for the population and pedigree association testing for quantitative data~\cite{Vis16}, and \emph{(b)}~conducted the bioinformatics analysis for~\cite{Joh15,AlM15,Gia16,Pui16}. \\
	 & 
	% &\textsc{Supervisor}: Dr. Mario Falchi
	}
	
	\dtlist{Jan 2012 - Dec 2013}{\textbf{Research associate} at the Department of Computer Science, University of Turin, Italy \\ %Recipient of a 2-year fellowship for her research proposal, awarded by the Italian Minister of Education, University and Research. \\
	 &\textsc{Research activity}: Alessia Visconti \emph{(a)}~developed and implemented a novel bi-clustering approach leveraging additional knowledge~\cite{Vis13a}, \emph{(b)}~contributed to the development of a novel exact algorithm for answering Maximum a Posteriori queries on tree structures~\cite{Esp13}, \emph{(c)}~applied machine learning approaches for the prediction and interpretation of the lipophilicity of small peptides~\cite{Vis15a}, and \emph{(d)}~conducted the bioinformatics analysis to measure the ability of more than 200 compounds of acting as hydrogen bond donors~\cite{Erm14}. \\
	 & 
	% &\textsc{Supervisors}: Dr. Roberto Esposito
	}
	\dtlist{Jun 2011 - Dec 2011}{\textbf{Visiting researcher} at the Center of Biological Sequence Analysis, Technical University of Denmark, Denmark 
	\\ % at the Regulatory Genomics Group at the Center of Biological Sequence Analysis, Systems Biology Department, Technical University of Denmark, Denmark 
	 &\textsc{Research activity}: Alessia Visconti \emph{(a)}~continued her PhD project developing a new method for the reverse engineering of gene regulatory networks that uses a popular econometrics statistical hypothesis test, namely the Granger Causality~\cite{Vis12b}, and \emph{(b)}~developed an \emph{ensemble} approach for the prediction of promoter activity. \\
	 & 
	% &\textsc{Supervisor}: Prof. Christopher Workman
	}

	\dtlist{Jan 2009 - Dec 2011}{\textbf{PhD student} at Department of Computer Science, University of Turin, Italy \\
	  &\textsc{Research activity}: Alessia Visconti developed and implemented: \emph{(a)}~a \emph{de novo} framework (accompanied by a web interface) performing protein motifs identification and allowing the simultaneous associations between groups of protein sequences and groups of motifs thanks to a constrained co-clustering approach~\cite{Cor09a}, \emph{(b)}~a new methodology (accompanied by a web interface) that provides meaningful co-clusters whose discovery and interpretation are enhanced by embedding gene ontology (GO) annotations~\cite{Vis11c,Cor09b}, \emph{(c)}~a novel algorithm for the rewriting of the GO aiming at obtaining a more compact and informative ontology~\cite{Vis11a,Vis10a}, \emph{(d)}~two algorithms for the reverse engineering of gene regulatory networks~\cite{Vis11b,Mar12,Vis12b}, and \emph{(e)}~contributed to a modular framework for the analysis of metagenomics sequences leading the co-clustering module development~\cite{Bon11}. \\
	  &
	  	% &\textsc{Supervisors}: Dr. Roberto Esposito, Prof. Marco Botta
	}

\end{doubletablelist}

\newpage

\begin{doubletablelist}
		
	\dtlist{Sep 2008 - Dec 2008}{\textbf{Research assistant} at the Department of Computer Science, University of Turin and in collaboration with the Department of Arboriculture and Pomology, University of Turin, Italy  \\
 	 &\textsc{Research activity}: During this post, Alessia Visconti contributed to the development of computational approaches for the classification and traceability of fruits produced in Piemonte. \\
	 &
% 	% &\textsc{Supervisors}: Prof. Marco Botta, Prof. Roberto Botta
 	}

 \end{doubletablelist}
 
 
 
 % --------------------------------
 % RESEARCH VISIT

 % \dtlist{Research Visits}{
 % 	\vspace{-0.6cm}
 % 	\begin{itemize} %[itemsep=-0.5ex]
 % 		\minusitem \begin{minipage}{0.65\textwidth}
 % 		 \textsc{Jan 2015} -- visiting researcher at Weill Cornell Medical College in Qatar
 % 		\end{minipage}
 % 		\minusitem \begin{minipage}{0.65\textwidth}
 % 		 \textsc{Jun - Dec 2011} -- visiting PhD student at the Regulatory Genomics Group, Center of Biological Sequence Analysis, Systems Biology Department, Technical University of Denmark, under the supervision of Prof. C. Workman
 % 		\end{minipage}
 % 	\end{itemize}
 % }